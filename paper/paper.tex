% vim: set spell spelllang=en tw=100 et sw=4 sts=4 :

\documentclass{llncs}

\usepackage{showframe}

\usepackage{microtype}
\usepackage{complexity}                % \P, \NP etc
\usepackage{tikz}                      % For pretty pictures
\usepackage{amsmath}                   % \operatorname
\usepackage{hyperref}                  % clicky links
\usepackage{cleveref}                  % no need to type Figure etc
\usepackage{gnuplot-lua-tikz}          % graphs
\usepackage[ruled,vlined]{algorithm2e} % algorithms (after cleverref!)
\usepackage{placeins}                  % \FloatBarrier
\usepackage{cite}

\usetikzlibrary{arrows, shadows, calc, positioning, decorations, decorations.pathreplacing, patterns}

% lncs style
\crefname{algocf}{Algorithm}{Algorithms}
\Crefname{algocf}{Algorithm}{Algorithms}
\crefname{figure}{Fig.}{Figs.}
\Crefname{figure}{Fig.}{Figs.}
\crefname{table}{Table}{Tables}
\Crefname{table}{Table}{Tables}
\crefname{proposition}{Proposition}{Propositions}
\Crefname{proposition}{Proposition}{Propositions}

\newcommand{\lineref}[1]{line~\ref{#1}}

\newcommand*\samethanks[1][\value{footnote}]{\footnotemark[#1]}

\title{Clique and Constraint Models for Maximum Common (Connected) Subgraph Problems}

\author{Ciaran McCreesh\thanks{This work was supported by the Engineering and Physical Sciences
    Research Council [grant number EP/K503058/1]}\inst{1} \and Samba Ndojh Ndiaye\thanks{This work
was supported by the ANR project SoLStiCe (ANR-13-BS02-0002-01)}\inst{2} \and Patrick
Prosser\inst{1} \and Christine Solnon\samethanks[2] \inst{3}}
\institute{University of Glasgow, Glasgow, Scotland \and
Universit\'e Lyon 1, LIRIS, UMR5205, F-69621, France  \and INSA-Lyon, LIRIS, UMR5205, F-69621, France}

\begin{document}

\maketitle

\begin{abstract}
    The maximum common subgraph problem is to find the largest subgraph common to two given graphs.
    This problem can be solved either by constraint-based search, or by reduction to the maximum
    clique problem. We evaluate these two approaches using modern algorithms. We also propose a
    restricted branching technique for the connected version of the problem, and show how to
    implement this in constraint-based and clique-inspired algorithms.
\end{abstract}

\section{Introduction}

Max common subgraph ?? cite some history and why we care

Connected \cite{DBLP:journals/tcs/Koch01,DBLP:conf/mco/VismaraV08}

CP models \cite{DBLP:conf/cp/NdiayeS11}

Clique models, discuss below.

\subsection{Definitions and Notation}

\subsection{Datasets}

\cite{DBLP:journals/prl/SantoFSV03,DBLP:journals/jgaa/ConteFV07}

\section{Re-Evaluating the Clique Model}

Previous experimental work has used a maximal clique enumeration algorithm, even for the
maximisation problem. ?? check this! I recall someone using a bad max clique algorithm too.
\cite{DBLP:conf/sspr/BunkeFGSV02,DBLP:journals/jgaa/ConteFV07}

We instead use a maximum clique algorithm. There has been a lot of recent
research on this
\cite{DBLP:conf/dmtcs/TomitaS03,DBLP:journals/jgo/TomitaK07,DBLP:conf/walcom/TomitaSHTW10,DBLP:journals/cor/SegundoRJ11,DBLP:journals/algorithms/Prosser12,DBLP:journals/ol/SegundoMRH13,DBLP:conf/ictai/LiFX13,DBLP:journals/cor/SegundoT14,DBLP:conf/lion/SegundoLB14,DBLP:conf/cp/McCreeshP14,DBLP:journals/jco/BatsynGMP14,DBLP:journals/cor/SegundoNB15,DBLP:conf/lion/NikolaevBS15,DBLP:conf/lion/LiJX15}.

Also sparse stuff, but association graphs aren't sparse enough.

Say which clique algorithm we're using.

\subsection{Reduction to Maximum Clique in an Association Graph}

Explain association graph. ?? cite history, connections to microstructure

Discuss how edge labels, vertex labels, directed edges, etc, are handled. Note that we can generally
only handle very local side constraints, or injectivity (which is decomposable).

\subsection{Experimental Evaluation}

\cref{figure:unconnected-33}

\cref{figure:unconnected-plain}

\begin{figure}[p]
    \centering
    \input{gen-graph-unconnected-33}
    \caption{Clique versus CP, 33\% labelled. On the top, the cumulative
    number of instances solved by each approach, in under a certain time. On the bottom, the best
    clique approach compared to the best CP approach: the colour at a point indicates the number of
    instances with these runtimes.} \label{figure:unconnected-33}
\end{figure}

\begin{figure}[p]
    \centering
    \input{gen-graph-unconnected-plain}
    \caption{Clique versus CP, undirected, unlabelled. On the top, the cumulative
    number of instances solved by each approach, in under a certain time. On the bottom, the best
    clique approach compared to the best CP approach: the colour at a point indicates the number of
    instances with these runtimes.} \label{figure:unconnected-plain}
\end{figure}

\subsection{Parallel Search}

\cite{DBLP:journals/jcc/KoncDTRJ12,DBLP:journals/algorithms/McCreeshP13,DBLP:journals/topc/McCreeshP15,DBLP:journals/cor/SegundoLP16}

?? Stuff on EHPs, how work stealing matters here too

\section{Maximum Common Connected Subgraph}

Connectedness is not hereditary, but can be built up one vertex at a time. Alternatively,
connectedness is a constraint.

\subsection{Connectedness via Restricted Branching}

The key idea is that whenever we branch (except for the first choice we make), we select only from
variables which correspond to vertices which are adjacent to at least one other variable which has
been instantiated to a non-null value. If no such variables remain, the subgraph we are building
cannot grow any further, and all remaining variables must be instantiated to null.

?? We only need to work off whichever input gives us the variables. But the CP approach seems to
look at both?

?? Patrick reckons this reminds him of Prim's algorithm for minimum spanning trees.

\subsection{Connectedness as a Constraint}

Connectedness \cite{Brown:2005}

Reachability \cite{DBLP:conf/cp/DoomsDD05,DBLP:conf/cp/QuesadaRD05}

\subsection{A Clique-Inspired Algorithm}

We can use the branching idea inside a ``maximum clique'' algorithm, by providing additional
information on the meaning behind vertices in the association graph as input. To fit in with the
bitset encoding, we preconstruct a bitset for each vertex $v$ in the association graph, saying which
association graph vertices correspond to vertices adjacent to $v$ in the left-hand graph.

?? This gets confusing. Maybe we can come up with some clever terminology: when I was talking about
search trees and clique algorithms, I used the term 'node' for something in a search tree, and
'vertex' for something in a graph. Maybe we can use different terms for things in the association
graph, versus things in the two input graphs? Should we call association graph vertices
'assignments' and edges 'compatibilities' instead?

We keep track of allowed branching vertices in a set we call $A$. We only branch on vertices which
are present in $P \cap A$.

?? We could make this faster, since it'll be the same for each vertex corresponding to the same
left-hand vertex.

We must also ensure that the bound remains valid. The branching and bounding are integrated, and it
is not possible to alter the branching rule without also changing the bound. Our observation is that
the bound rule works for any valid colouring, not just a greedy one, so long as we always iterate
from the last colour class downwards, without skipping any vertices. Thus we construct a two-pass
greedy colouring such that all the vertices in $P \setminus A$ are to the left, and the vertices in
$P \cap A$ are to the right.

\cite{DBLP:conf/cp/McCreeshP14}

\subsection{Experimental Evaluation}

\paragraph{Connected, Undirected, 33\% Labelled} \cref{figure:connected-undir33} Above two seconds,
    the clique approach dominates except for three instances, and there is only one instance which
    the CP model can solve but the clique model cannot (?? is this an EHP and does parallel get rid
    of it?). The clique model is unable to solve only ??  instances within one hour, ?? all of which
    have 90 or 100 vertices.  On the other hand, for trivial instances, the clique approach is often
    expensive due to the association graph construction.

\paragraph{Connected, Undirected, Unlabelled} \cref{figure:connected-plain}, looks like the clique
approach doesn't do so well\ldots

?? My gut feeling is that if we have edge labels, the clique model wins, and if we don't, it's
likely to lose. So maybe I should run some experiments with vertex labels but not edge labels to
test this.

\begin{figure}[p]
    \centering
    \input{gen-graph-connected-undir33}
    \caption{Clique versus CP, connected, undirected, 33\% labelled, up to 100 vertices. On the top,
        the cumulative number of instances solved by each approach, in under a certain time. On the
        bottom, the best clique approach compared to the best CP approach: the colour at a point
        indicates the number of instances with these runtimes.} \label{figure:connected-undir33}
\end{figure}

\begin{figure}[p]
    \centering
    \input{gen-graph-connected-plain}
    \caption{Clique versus CP, connected, undirected, unlabelled, up to 25 vertices (and some 30 and
        35, which are still running\ldots). On the top, the cumulative number of instances solved by
    each approach, in under a certain time. On the bottom, the best clique approach compared to the
best CP approach: the colour at a point indicates the number of instances with these runtimes.}
\label{figure:connected-plain}
\end{figure}

\subsection{Parallel Search}

\cite{DBLP:conf/ictai/MinotNS15}

\section{Conclusion}

Some immoral perspectives on closely coupling branching with a particular constraint. Whether this
is useful for other things, like ASP.

We haven't discussed partial subgraphs, or connected directed.

Possibly portfolios.

Better instances.

\FloatBarrier

\bibliographystyle{splncs}
\bibliography{paper}

\end{document}

